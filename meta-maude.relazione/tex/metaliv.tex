\section{Il meta-livello di Maude}

La logica di \emph{rewriting} di \emph{Maude} è riflessiva \cite{maudemanual},
ovvero descritta in termini di se stessa tramite una teoria universale. 
Rendendo disponibile all'utente l'accesso alla teoria universale di Maude, è
possibile ragionare ponendosi al meta-livello sia della \emph{``struttura''} dei
moduli e termini \emph{Maude}, sia delle strategie risolutive dei procedimenti
di inferenza, riduzioni, matching e ricerche.

La \emph{riflessione} della logica di rewriting può essere descritta attraverso
la relazione :

\[
	\mathcal{R} \vdash t \to t' \Leftrightarrow \mathcal{U} \vdash \langle
	\bar{\mathcal{R}}, \bar{t} \rangle \to \langle \bar{\mathcal{R}'},	\bar{t'}
	\rangle
\]

Dove $\mathcal{R}$ è una teoria di \emph{rewrite},  $\mathcal{U}$ è la teoria
di rewrite universale, in grado di rappresentare qualunque teoria di rewrite
finita in termini di se stessa. Il simbolo $\to$ indica che il termine a
sinistra può essere riscritto come il termine di destra; $t, t'$ sono termini,
$\langle \bar{\mathcal{R}}, \bar{t} \rangle$ è un termine che rappresenta
secondo la teoria $\mathcal{U}$ la coppia $(\mathcal{R}, t)$.

La teoria $\mathcal{U}$ è universale e può essere espressa anche in termini di
se stessa. Dunque la relazione precedente crea una ``reflective tower'' nella
quale è possibile muoversi sia alto che in basso. Il livello più basso nella
torre è chiamato forma canonica.

La teoria universale di rewrite $\mathcal{U}$ è implementata in
Maude nel modulo \texttt{META-LEVEL}, il quale, oltre a contenere operazioni
per muoversi lungo la reflective tower, contiene sort per descrivere al
meta-livello moduli e termini.

La maggior parte delle operazioni del meta-livello di Maude
sono parziali, poiché possono esistere argomenti del Sort corretto che tuttavia
non sono una corretta meta-rappresentazione o non esiste una valido risultato.

\subsection{Accesso dinamico a moduli e termini}

Termini e moduli Maude possono essere rappresentati come altri termini Maude,
sruttando la teoria di rewriting universale, permettendo di caricare moduli
dinamicamente, lavorare e costruire termini di Sort anche sconosciuti ed
eseguire su tali meta-moduli e meta-termini delle operazioni di reduce e rewrite.

\subsubsection{upTerm e downTerm}

Dato un qualunque termine Maude, è possibile salire e scendere la
\emph{reflectio tower} utilizzando le operazioni \emph{upTerm} e
\emph{downTerm}:
\begin{Verbatim}[fontsize=\small]
op upTerm : Universal -> Term [poly (1) special (...)] .
op downTerm : Term Universal -> Universal [poly (2 0) special (...)] .
\end{Verbatim}
\emph{upTerm} prende in ingresso un qualunque termine $t$ e restituisce sempre
la sua meta-rappresentazione $\bar{t}$ di sort \emph{Term}. \emph{downTerm}
prende in ingresso $\bar{t}$, meta-rappresentazione di un termine $t$, ed un secondo argomento di
un certo kind. Se $t$ è dello stesso kind del secondo argomento, la
\emph{downTerm} restituisce $t$, altrimenti restituisce il secondo argomento.
\begin{Verbatim}[fontsize=\small]
Maude> red in META-LEVEL : upTerm ( 'a ) .
result Constant: ''a.Sort

Maude> red in META-LEVEL : upTerm( 'a 'b ) .
result GroundTerm: '__[''a.Sort,''b.Sort]

Maude> red in META-LEVEL : downTerm ( ''a.Sort , ('prova).Qid ) .
result Sort: 'a

Maude> red in META-LEVEL : downTerm ( ''a.Sort , ("Fail").String ) .
result String: "Fail"
\end{Verbatim}

\subsubsection{upModule}
\label{sec:upMod}
I comandi \emph{mod} e \emph{fmod} permettono di definire moduli che Maude
caricherà nel proprio database interno. Quando si eseguono istruzioni come
\begin{Verbatim}[fontsize=\small]
Maude> reduce in MIO-MODULO : prova( 1 , 2 , 3 ) .
\end{Verbatim}
Maude accede al database, cerca il modulo \emph{MIO-MODULO} ed effettua la
\emph{reduce} dell'espressione \emph{prova(1,2,3)}: l'accesso al database è
statico.

Il meta-livello di Maude permette di ottenere la meta-rappresentazione di un
modulo precaricato nel database, partendo dal suo nome espresso con un quoted
identifier, tramite l'operazione:
\begin{Verbatim}[fontsize=\small]
op upModule : Qid Bool ~> Module [special (...)] .
\end{Verbatim}
Il primo argomento è il nome del modulo, il secondo argomento è true se si
desidera ottenere anche la meta-rappresentazione di tutte le operazioni
derivate dai moduli importati, false altrimenti. Il valore di ritorno Module è
un particolare sort che meta-rappresenta un modulo Maude; ha una struttura
definita, con un ordine preciso, dunque è semplice da analizzare ed alterare, ma
non è possibile caricare un meta-modulo nel database di Maude.

\subsubsection{metaRewrite e metaApply}
\label{sec:mapply}
Dato un oggetto di tipo Module è possibile eseguire dinamicamente operazioni
di reduce, rewrite e apply (applicazione di una rule data una regola) partendo
dalla meta-rappresentazione del termine da ridurre o riscrivere. L'operazione
\emph{metaRewrite} permette di eseguire su un meta-modulo \emph{Module} la
rewrite di un meta-termine \emph{Term} fino ad un numero di volte specificato da
\emph{Bound}.
\begin{Verbatim}[fontsize=\small]
sort Bound .
subsort Nat < Bound .
op unbounded :-> Bound [ctor] .
op metaRewrite : Module Term Bound ~> ResultPair [special (...)] .
\end{Verbatim}
La metaRewrite ritorna valore di sort ResultPair:
\begin{Verbatim}[fontsize=\small]
sort ResultPair .
op {_,_} : Term Type -> ResultPair [ctor] .

op getTerm : ResultPair -> Term .
op getType : ResultPair -> Type .
\end{Verbatim}
contenente la meta-rappresentazione del risultato e la meta-rappresentazione del
suo kind o sort. La funzione \emph{metaRewrite} è parziale, quindi se Module o
Term sono sintatticamente corretti, ma non sono corrette meta-rappresentazioni,
è restituito un risultato indefinito di kind [ResultPair] .

L'operazione \emph{metaApply} (e la sua generalizzazione \emph{metaXapply})
permettono di applicare una rule basandosi sulla label ed ottenere un risultato dettagliato sulla
riscrittura. La chiamata di $metaXapply(\bar{\mathcal{R}}, \bar{t}, \bar{l},
\sigma, n, b, m)$ riduce $t$ con le equazioni di $\mathcal{R}$, dopodiché crea
una sequenza di soluzioni applicando tutte le rule chiamate $l$ al termine $t$ e
scarta i primi $n$ risultati. $b$ e $m$ specificano il livello di nesting al
quale applicare la riscrittura (per una riscrittura non limitata: 0 -
$unbounded$).
\begin{Verbatim}[fontsize=\small]
sorts Result4Tuple Result4Tuple? .
subsort Result4Tuple < Result4Tuple? .
op {_,_,_,_} : Term Type Substitution Context -> Result4Tuple [ctor] .
op failure : -> Result4Tuple? [ctor] .

op metaXapply : Module Term Qid Substitution Nat Bound Nat ~>
                                                      Result4Tuple? .
\end{Verbatim}
Se l'operazione è andata a buon fine, il risultato \emph{Result4Tuple} contiene:
\begin{description}
  \item[Term:]meta-termine riscritto.
  \item[Type:]meta-tipo del termine riscritto.
  \item[Substitution:]dettagli sulla sostituzione effettuata dalla apply.
  \item[Context:]contesto nel quale è avvenuta la riscrittura. Contiene una
  lista di meta-termini non riscritti e un singolo ``hole'' dove è avvenuta la
  riscrittura.
\end{description}

\subsubsection{metaMatch}
Un'altra potente applicazione del meta-livello di Maude è la metaMatch (e la
sua generalizzazione metaXmatch), che permette di eseguire un match fra due
meta-termini all'interno di un meta-modulo permettendo di specificare condizioni
booleane e di enumerare le soluzioni.
\begin{Verbatim}[fontsize=\small]
sorts MatchPair MatchPair? .
subsort MatchPair < MatchPair? .
op {_,_} : Substitution Context -> MatchPair [ctor] .
op noMatch : -> MatchPair? [ctor] .
op metaXmatch : Module Term Term Condition Nat Bound Nat ~>
                                             MatchPair? [special (...)] .
\end{Verbatim}
Il risultato \emph{MatchPair} contiene le sostituzioni effettuate e il contesto
nel quale è avvenuta la sostituzione.
