\section{Progetto PCTL}

La seconda parte del progetto riguarda l'implementazione di un verificatore di
proprietà PCTL utilizzando APMC.

\subsection{Strategia risolutiva}
\label{sec:strategia}
Il nuovo sistema, il verificatore, segue l'algoritmo descritto
nella sezione \ref{sec:APMC}. L'appossimazione $\epsilon$, il fattore di
confidenza $\delta$ e la profondità massima delle path sono forniti dall'utente
insieme allo stato iniziale, al modello e alla formula da verificare. 

Per la generazione della path casuale, sfrutto il simulatore CTMC
realizzato nella prima parte del progetto, ma anziché generare interamente una
path di profondità massima, simulo il sistema un passo alla volta, valutando la
formula in ogni nuovo stato.

In generale, per risolvere una formula seguo questo procedimento:
\begin{enumerate}
  \item Se la formula $f \in \phi$: risolvo la formula per lo
  stato corrente.
  \item Se la formula $f \in \psi , f = X \phi$: nello stato successivo
  verifico $\phi$.
  \item Se la formula $f \in \psi , f = \phi U \phi'$: se vale $\phi'$, $f$ è
  vera, altrimenti se vale $\phi$, nello stato successivo verifico $\phi U
  \phi'$. Se non vale né $\phi$ né $\phi'$, $f$ non è vera.
  \item Se la formula $f \in \psi , f = \phi U^{\leq\tau} \phi'$: se vale
  $\phi'$, $f$ è vera, alrimenti se vale $\phi$, nello stato successivo verifico
  $\phi U^{\leq\tau} \phi$. Se non vale né $\phi$ né $\phi'$, $f$ non è vera. Se
  arrivo in uno stato in cui il tempo $t > \tau$, la formula $f$ non è vera.
\end{enumerate}


\subsection{Sintassi delle Formule}
All'interno del modulo \emph{PCTL}, ho definito i sort e i costruttori necessari
a definire le formule PCTL. Come predicato $p$, utilizzo meta-stato \emph{Term},
con semantica $p \equiv S$, dove $S$ è il meta-stato corrente, oppure
$p(State, Condition)$ in grado di definire predicati contenenti stati che
effettuano un match con lo stato corrente e che devono rispettare certe
condizioni con sintassi definita dal sort \emph{Condition} \cite{maudemanual}.

\begin{Verbatim}[fontsize=\small]
*** Formula -> phi ; PathFormula -> psi
sort Formula PathFormula .

*** Un predicato p è un meta-stato.
subsort Term < Formula .
*** p(Term, Condition) effettua un matching di Term con lo stato corrente,
*** controllando la Condition.
op p : Term Condition -> Formula [ ctor ] .
op FALSE : -> Formula [ ctor ] .
op TRUE : -> Formula [ ctor ] .
op _ implies _ : Formula Formula -> Formula [ ctor prec 90 ] .
op _ and _ : Formula Formula -> Formula [ ctor prec 80 ] .
op _ or _ : Formula Formula -> Formula [ ctor prec 80 ] .
op not _ : Formula -> Formula [ ctor prec 75 ] .
op P[_](_) : Float PathFormula -> Formula [ ctor prec 70] .

op (_)U[_](_) : Formula Float Formula -> PathFormula [ ctor prec 92] .
op (_)U(_) : Formula Formula -> PathFormula [ ctor prec 92] .
op X_ : Formula -> PathFormula [ ctor ] .
*** Per risolvere ricorsivamente X phi, creo un altro operatore C phi:
*** true se phi vale nello stato corrente
op C_ : Formula -> PathFormula [ ctor ] .
\end{Verbatim}

Il costruttore \emph{C\_} è un artefatto che ho introdotto per semplificare la
risoluzione ricorsiva delle \emph{PathFormula} ($\psi$). Il significato
semantico di $C\phi$ è ``nello stato corrente deve valere $\phi$''.
L'operatore \emph{P[\_](\_)} rappresenta l'operatore di probabilità e il suo
significato è : $Prob_{\geq p}(\psi)$.

L'operatore $p(S1, C)$ è verificato se , con stato corrente $S$, esiste un
matching fra $S1$ e $S$ che rispetti la condizione $C$. Questa procedura è
facilmente implementabile attraverso la \emph{metaXmatch} imponendo che il
Context sia vuoto, ovvero che non esistano termini senza matching.

 \begin{Verbatim}[fontsize=\small]
***
*** match(M, t1, t2, tc) verifica che esista un matching fra il meta-stato.
*** t1 e l'intero meta-stato t2, rispettando la condizione tc.
***   - Module : meta-modulo utente.
***   - Term : meta-stato da verificare.
***   - Term : meta-stato corrente.
***   - Condition : condizione che il meta-stato da verificare deve rispettare.
op match : Module Term Term Condition -> Bool .
\end{Verbatim}

\subsection{Verifica di Proprietà}

Per verificare una formula su un modello a partire da un certo stato
iniziale, si utilizza l'operazione \emph{verify}, specificando il nome del
modulo utente, il meta-stato iniziale, la formula e tutti i parametri necessari.
L'operazione \emph{verify} funge da wrapper per una operazione \emph{verify}
con differenti argomenti, che verifica la formula.

\begin{Verbatim}[fontsize=\small]
***
*** Verfica la formula data (entry point)
***   - Qid : Nome del modulo utente su cui c'è il modello .
***   - Formula : formula PCTL fa verificare.
***   - Term : meta-stato iniziale
***   - Float : approximation.
***   - Float : Confidence.
***   - Nat : Max Depth.
***   - Nat : indice di partenza per la generazione di numeri casuali.
op verify : Qid Formula Term Float Float Nat Nat -> Bool .
	
***
*** Verfica la formula data
***   - Module : meta-modulo utente.
***   - Formula : formula PCTL fa verificare.
***   - SimulationResult : meta-stato corrente + tempo + deadlock?
***   - Float : approximation.
***   - Float : Confidence.
***   - Nat : Max Depth.
***   - Nat : indice di partenza per la generazione di numeri casuali.
op verify : Module Formula SimulationResult Float Float Nat Nat -> Bool .
\end{Verbatim}

L'operazione verify, verifica le formule $\phi$. Quando compare una formula del
tipo \emph{P[\_]($\psi$)}, è computato il numero di sample $N$ da
effettuare e verify richiama l'operazione \emph{approxChecking} che per $N$ volte crea una
path random verificando la formula $\psi$. Come ulteriore miglioramento delle
prestazioni, interrompo il model checking se la probabilità incrementale ha già
raggiunto la probabilità target (le formule sono monotone, dunque la
probabilità può solo aumentare ulteriormente).
	
\begin{Verbatim}[fontsize=\small]
***
*** Esegue un model checking approssimato della PathFormula data
***   - Module : meta-modulo utente. 
***   - PathFormula : path-formula PCTL fa verificare.
***   - SimulationResult : Stato corrente + tempo.
***   - Nat : Numero di campioni corretti.
***   - Nat : Numero di campioni testati.
***   - Nat : Numero di campioni massimi.
***   - Float : Probabilità da raggiungere.
***   - Float : approximation.
***   - Float : Confidence.
***   - Nat : Max Depth.
***   - Nat : indice di partenza per la generazione di numeri casuali. 
op approxChecking : Module PathFormula SimulationResult
                    Nat Nat Nat Float Float Float Nat Nat -> Float  .
\end{Verbatim}

Per verificare una formula all'interno durante la generazione di una path, ho
creato una operazione \emph{checkOnRandomPath} che ricorsivamente incrementa di
un passo la simulazione, verificando di volta in volta $\psi$, seguendo le
strategie descritte nella sezione \ref{sec:strategia}:
\begin{description}
  \item[$\phi_1 U \phi_2$:]nello stato corrente verifico $\phi_2$,
  se è vera, allora la formula è verificata, altrimenti verifico $\phi_1$. Se
  $\phi_1$ è false, allora la formula è falsa, altrimenti avanzo la simulazione
  di un passo con l'operazione \emph{simNcheck}, verificando nello stato
  successivo la stessa formula $\phi_1 U \phi_2$.
  \item[$\phi_1 U^{\leq \tau} \phi_2$:] si comporta esattamente come la versione
  non bounded, ma prima di tutto controllo se $\tau \geq t$, dove $t$ è il tempo
  nello stato corrente. Se la relazione è falsa, la formula non è verificata.
  \item[$X \phi$:]avanzo la simulazione di un passo con l'operazione
  \emph{simNcheck}, verificando nello stato successivo la formula $C \phi$.
  \item[$C \phi$:]controllo nello stato corrente se $\phi$ è verificata.
\end{description}

Ovviamente la verifica di $\phi$ avviene richiamando l'operazione \emph{verify}.
\begin{Verbatim}[fontsize=\small]
***
*** Controlla una PathFormula su una path casuale
***   - Module : meta-modulo utente.
***   - PathFormula : path-formula PCTL da verificare.
***   - SimulationResult : Stato corrente + tempo.
***   - Nat : Profondità corrente.
***   - Nat : Profondità massima.
***   - Float : approximation.
***   - Float : Confidence.
***   - Nat : indice di partenza per la generazione di numeri casuali.
op checkOnRandomPath : Module PathFormula SimulationResult
                             Nat Nat Float Float Nat -> Bool .
                             
***
*** Esegue una simulazione di un passo e controlla la PathFormula
*** sul nuovo stato.
***   - Module : meta-modulo utente
***   - PathFormula : path-formula PCTL da verificare nello
***                                              stato successivo.
***   - SimulationResult : Stato corrente + tempo
***   - Nat : Profondità corrente
***   - Nat : Profondità massima
***   - Float : approximation
***   - Float : Confidence
***   - Nat : indice di partenza per la generazione di numeri casuali
op simNcheck : Module PathFormula SimulationResult
                              Nat Nat Float Float Nat -> Bool .

\end{Verbatim}

\subsection{Confronto con PRISM}

Se con il framework realizzato si può avere una maggiore espressività
rispetto a PRISM, lo si paga enormemente nelle prestazioni. Di seguito riporto
dei test che ho effettuato su tre moduli diversi, descritti nella sezione
\ref{sec:esempio}. PRISM mette a disposizione l'operatore $P=?[ \ldots ]$, che
ritorna la probabilità anziché un booleano. Per ottenere lo stesso risultato nel
framework, basta valutare $P[1.0]( \ldots )$ e abilitare le print con il comando
\begin{Verbatim}
Maude> set print attribute on .
\end{Verbatim}
Al termine della valutazione è stampato il valore di probabilità finale.

\subsubsection{Modello readers-writers}

Il modello readers-writers è descritto nella sezione \ref{sec:rw}. Di seguito
riporto anche la sua implementazione in PRISM:
\begin{Verbatim}[fontsize=\small]
ctmc

// Numero di processi
const int N = 100;

module RW
start : [0..N] init N;
choise : [0..N] init 0;
waitr : [0..N] init 0;
waitw : [0..N] init 0;
sem : [0..N] init 1;
read : [0..N] init 0;
write : [0..N] init 0;

[t1] start>0 & choise<N  -> start :
     (start'=start-1)&(choise'=choise+1);
[t2] choise>0 & waitr<N & waitw<N -> choise :
     (choise'=choise-1) & (waitr'=waitr+1);
[t3] choise>0 & waitr<N & waitw<N -> choise :
     (choise'=choise-1) & (waitw'=waitw+1);
[t4] waitr>0 & sem>0 & read<N -> waitr*sem :
     (waitr'=waitr-1) & (read'=read+1);
[t5] waitw>0 & sem>0 & read=0 & write<N -> waitw*sem :
     (waitw'=waitw-1) & (sem'=sem-1) & (write'=write+1);
[t6] read>0 & start<N -> read :
     (read'=read-1) & (start'=start+1);
[t7] write>0 & sem<N & start<N -> write :
     (write'=write-1) & (start'=start+1) & (sem'=sem+1);

endmodule

// Definizione dei rate

const double r1=10;
const double r2=5;
const double r3=5;
const double r4=20;
const double r5=40;
const double r6=50;
const double r7=50;

module base_rate 
[t1] true  -> r1 : true;
[t2] true  -> r2 : true;
[t3] true  -> r3 : true;
[t4] true  -> r4 : true;
[t5] true  -> r5 : true;
[t6] true  -> r6 : true;
[t7] true  -> r7 : true;
endmodule
\end{Verbatim}

Il test è stato fatto con condizione iniziale 100 processi, 10000 profondità
massima, $\epsilon = 0.01$ e $\delta = 1.0\cdot10^{-10}$.

\textbf{P=?[true U[0.1] waitw $>$ 0 $\&$ read $>$ 0]} (la probabilità che entro
0.1 secondi, ci sia un processo in lettura ed uno in attesa di scrivere).
La formula espressa nel framework è:
\begin{Verbatim}
red in CTMC-RW_VERIFIER : verify(100,
 P[1.0] ( (TRUE) U[0.1] ( 
    p( upTerm(< v("read", N:Nat) v("waitw", M:Nat) V:VarList >),
                  upTerm(N:Nat > 0 and M:Nat > 0) = 'true.Bool )))
   , 0.01, 0.0000000001, 10000, 42793) .
\end{Verbatim}
Dove CTMC-RW\_VERIFIER è un wrapper per facilitare la verifica delle operazioni
di verifica sul modulo CTMC-READER\_WRITER.

\begin{description}
	\item[PRISM:] tempo: 17.855 secondi . Probabilità: 0.7882288460727687
	\item[ENGINE:] tempo: 2036.773 secondi . Probabilità: 0.77615413803280076
\end{description}

La differenza di probabilità è 0.012074708, quindi rientra nei limiti
dell'approssimazione.

\subsubsection{Modello BSF}

Il modello del test della Bromosulftaleina è descritto nella sezione
\ref{sec:bsf}. Di seguito riporto anche la sua implementazione in PRISM:
\begin{Verbatim}[fontsize=\small]
ctmc
// Volume Plasma
const int VP = 1000;
// Volume Fegato
const int VF = 800;
// Colorante iniziale
const int IC = 500;

module BSF
Pl : [0..VP] init IC;
Fe : [0..VF] init 0;

[t1] Pl > 0 & Fe < VF -> Pl / VP :
      (Pl'=Pl-1) & (Fe'=Fe+1);
[t2] Fe > 0 & Pl < VP -> Fe / VF :
     (Pl'=Pl+1) & (Fe'=Fe-1);
[t3] Fe > 0 -> Fe / VF :
     (Fe'=Fe-1);

endmodule

//base rates

const double r1=0.454;
const double r2=0.698;
const double r3=0.432;

module base_rate 
[t1] true  -> r1 : true;
[t2] true  -> r2 : true;
[t3] true  -> r3 : true;
endmodule
\end{Verbatim}


Il test è stato fatto con condizione iniziale 800 volume fegato, 1000 volume
plasma, 500 colorante, 10000 profondità massima , $\epsilon = 0.05$ e
$\delta = 1.0\cdot10^{-4}$. \textbf{P=?[true U[50000] Fe = 0 $\&$ Pl = 0]}.
La formula espressa nel framework è:
\begin{Verbatim}
red in CTMC-BSF_VERIFIER : verify(1000, 800, 500,
 P[1.0] ( (TRUE) U[50000.0] ( 
    upTerm(< v("plasma", 0) v("fegato", 0) V:VarList >)))
   , 0.05, 0.0001, 10000, 1351) .
\end{Verbatim}
Dove CTMC-BSF\_VERIFIER è un wrapper per facilitare la verifica delle operazioni
di verifica sul modulo CTMC-BSF.

\begin{description}
	\item[PRISM:] tempo: 6.874 secondi . Probabilità: 0.6431095406360424
	\item[ENGINE:] tempo: 984.625 secondi . Probabilità: 0.69863705199394244
\end{description}

La differenza di probabilità è 0.055527511, quindi rientra nei limiti
dell'approssimazione.

\subsubsection{Modello MMCK}

Il modello del test M/M/C/K è descritto nella sezione
\ref{sec:mmck}. Di seguito riporto anche la sua implementazione in PRISM:
\begin{Verbatim}[fontsize=\small]
ctmc
// Numero di server (C)
const int nServ = 5;
// Dimension del buffer (K)
const int buffSize = 10;

const int lostSize = 1000;

module MMCK
Buff : [0..buffSize] init 0;
Losts : [0..lostSize] init 0;

[t1] Buff < buffSize ->
     (Buff'=Buff+1);
[t2] Buff >= buffSize & Losts < lostSize ->
     (Losts'=Losts+1);
[t3] Buff > 0 & Buff < nServ -> Buff :
     (Buff'=Buff-1);
[t4] Buff >= nServ -> nServ : 
     (Buff'=Buff-1);

endmodule

//base rates

const double r1=0.535;
const double r2=0.120;

module base_rate 
[t1] true  -> r1 : true;
[t2] true  -> r1 : true;
[t3] true  -> r2 : true;
[t4] true  -> r2 : true;
endmodule
\end{Verbatim}

Il test è stato fatto con condizione iniziale 10 dimensione buffer, 5 server,
profondità massima $K = 10000$, $\epsilon = 0.01$ e $\delta = 1.0\cdot10^{-10}$.

\textbf{P=?[true U[70.0] losts $>$ 0]} (la probabilità che il buffer regga
almeno 50 secondi). La formula espressa nel framework è:
\begin{Verbatim}
red in CTMC-MMCK_VERIFIER : verify(10, 5,
 P[1.0] ( (TRUE) U[70.0] ( 
    upTerm(< v("losts", 1) V:VarList >)))
   , 0.01, 0.0000000001, 10000, 12523) .
\end{Verbatim}
Dove CTMC-MMCK\_VERIFIER è un wrapper per facilitare la verifica delle
operazioni di verifica sul modulo CTMC-MMCK.

\begin{description}
	\item[PRISM:] tempo: 6.689 secondi . Probabilità: 0.41349972595809265
	\item[ENGINE:] tempo: 1227.478 secondi . Probabilità: 0.40806948016358197
\end{description}

La differenza di probabilità è  0.005430246, quindi rientra nei limiti
dell'approssimazione.
