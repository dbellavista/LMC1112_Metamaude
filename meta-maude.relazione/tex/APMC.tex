\section{Approximate Probabilistic Model Checking}
\label{sec:APMC}
La PCTL (\emph{Probabilistic Computation Tree Logic}), permette di esprimere
proprietà che si avverano con una probabilità $\rho \in J$ su di un particolare
modello:
 \[
\phi ::= p | FALSE | \phi \rightarrow \phi' | P_J[\psi]
\]
\[
\psi ::= X \phi | \phi U \phi' | \phi U^I \phi'
\]

Dato un modello CTMC, la verifica di proprietà PCTL è un poblema
computazionalmente difficile a causa dell'esplosione degli stati da
verificare. Una prima semplificazione si può ottenere considerando solo
formule \emph{monotone}, in grado di ridurre lo spazio degli stati per
particolari proprietà. Una formula $\psi$ è monotona se e solo se \[\exists j
\in \mathbb{N}, \forall \pi{}[j..], \pi{}[j..]\models \phi \wedge \forall i \in
\mathbb{N}, i > j , \pi{}[i..]\models \phi \] 
Ovvero, una formula è monotona se la validità su di uno stato implica la
validità su tutte le path che partono da esso. Grazie a questa proprietà, la
verifica di $Prob_{\geq P}[ \psi ]$ può essere limitata alla verifica
$Prob_{\geq P}[ \psi ]_k$, ovvero limitata a path con profondità pari a $k$,
poiché la probabilità che la formula $\psi$ sia vera, può solo aumentare.
Formule composte unicamente da $FALSE$, proposizioni $p$, implicazioni $\phi
\rightarrow \phi{}'$ e operatori $X \phi$, $\phi U \phi{}'$ comprese le versioni bounded, sono formule monotone. Una grammatica per le formule monotone è la seguente:
 \[
\phi ::= p | FALSE | \phi \rightarrow \phi' | P_{\geq \rho}[\psi]
\]
\[
\psi ::= X \phi | \phi U \phi' | \phi U^I \phi'
\]

Le formule monotone non risolvono il problema della difficoltà
computazionale, ma rendono possibile applicare un procedimento approssimato,
ottenendo risultati appartenenti ad un intorno del vero risultato con una certa confidenza.

Il calcolo della probabilità si può approssimare con la tecnica di \emph{fully
polynomial randomized approximation scheme (FPRAS)} \cite{DBLP:conf/vmcai/2004}.
Un FPRAS è un algoritmo $A(x, \epsilon, \delta, k), \epsilon \in [0, 1], \delta
\in [0, 1]$ il quale, dato in input lo stato iniziale del sistema, una approssimazione, un parametro di confidenza e la lunghezza delle path
ritorna un valore di probabilità tale che:
\[
	Prob[ \mid A(x,\epsilon,\delta, k) - \mu(x) \mid \leq \epsilon] \geq 1 - \delta
\]
dove $\mu(x)$ è il risultato esatto della formula analizzata.

Una implementazione di $A$ è la seguente \cite{HLMP04}:

\vspace{0.3cm}
\fbox{%
        \parbox{0.7\linewidth}{%
A($x, \epsilon, \delta$) :\newline
$N := \log{\frac{2}{\delta}}/2\epsilon^2$\newline
$A := 0$\newline
for $i \in [1 ; N]$ do\newline
 1. Genera una path casuale $\pi$, di lunghezza $k$.\newline
 2. Se la formula $\phi$ da valutare è vera, $A := A + 1$.\newline 
$Return A/N$
      }%
}
\vspace{0.3cm}

L'algoritmo ha complessità polinomiale in $\frac{1}{\epsilon}$ e
$\log{\frac{1}{\delta}}$.
